\documentclass{SUFEThesis}


\sufesetup{
    title = {上海财经大学 \LaTeX 模板},
    author = {白露佳},
    id = {2016110226},
    school ={统计与管理学院},
    discipline = {统计学},
    supervisor = {柏杨}
} 

\begin{document}
%%%%定义一级小标题(section),二级小标题(subsection),三级小标题(subsubsection)字体字号, 四级"(1)"还未定义
\titleformat

\maketitle

% !TeX root = ../main.tex
\begin{abstract}
论文的摘要是对论文研究内容和成果的高度概括。摘要应对论文所研究的问题及其研究目
的进行描述,对研究方法和过程进行简单介绍,对研究成果和所得结论进行概括。摘要应
具有独立性和自明性,其内容应包含与论文全文同等量的主要信息。使读者即使不阅读全
文,通过摘要就能了解论文的总体内容和主要成果。摘要以浓缩的形式概括研究课题的内容,中文摘要在300汉字左右,英文摘要应与中文摘要基本相对应。

本文介绍上海财经大学论文模板SUFEThesis的使用方法。本模板符合学校的本科论文格式要求。

本文的创新点主要有:
\begin{itemize}
  \item 简单易懂代码少
  \item 用2016本科毕业论文要求里的内容填充无关紧要的部分;
  \item 一边学习摸索一边编写新代码。
\end{itemize}

关键词是为了文献标引工作、用以表示全文主要内容信息的单词或术语。关键词不超过 6
个,每个关键词中间用分号分隔。(模板作者注:关键词分隔符不用考虑,模板会自动处
理。英文关键词同理。)
\sufesetup{
keywords = {毕业论文, 排版, \LaTeX, \TeX, CJK},
}
\end{abstract} 


\begin{abstract*}
An abstract of a dissertation is a summary and extraction of research work
and contributions. Included in an abstract should be description of research
topic and research objective, brief introduction to methodology and research
process, and summarization of conclusion and contributions of the
research. An abstract should be characterized by independence and clarity and
carry identical information with the dissertation. It should be such that the
general idea and major contributions of the dissertation are conveyed without
reading the dissertation.

An abstract should be concise and to the point. It is a misunderstanding to
make an abstract an outline of the dissertation and words ``the first
chapter'', ``the second chapter'' and the like should be avoided in the
abstract.

Key words are terms used in a dissertation for indexing, reflecting core
information of the dissertation. An abstract may contain a maximum of 5 key
words, with semi-colons used in between to separate one another.
\sufesetup{
keywords* = {thesis, template, \LaTeX, \TeX, CJK },
}
\end{abstract*}

\newpage
\tableofcontents
\newpage
\setcounter{page}{1}%开始正文第一页

\section{序言}
序言应说明本课题的意义、目的、主要研究内容、范围及应解决的问题。
\subsection{研究背景及意义}

\subsection{研究现状与评价}


\subsection{本文贡献与创新点}

\subsection{文章基本思路与结构}


\section{论文格式概述}

\subsection{正文之前}

\subsubsection{封页}

封页上的内容一律按照统一封面的样张式样打印,必须正 确无误。题目用二号黑体字,其他用四号宋体字。

\subsubsection{题目和摘要}

论文(设计)题目为三号黑体字,可以分为1或2行 居中打印。论文(设计)题目下空一项居中打印“摘要”二字(三号 黑体),字间空一格,标点符号占一格。“摘要”二字下空一行打印 内容(四号宋体)。每段开头空二格,标点符号占一格。摘要内容后 下空一行打印“关键字”三字(四号黑体字),其后为关键词(四号 宋体)。英文摘要题目全部采用Arial字体,可分成1-3行居中打印, 以下均用五号Arial字体。摘要以浓缩的形式概括研究课题的内容,中文摘要在 300汉字左右,英文摘要应与中文摘要基本相对应。关键词是表述论文(设计)主题内容信息的单词或 术语,关键词数量一般不超过6个。每一个关键词之间用逗号隔 开,最后一个关键词后不用标点符号。

\subsubsection{目录}
“目录”二字(三号黑体),下空二行为章、节、小节及其开始页码。目录是论文(设计)各组成部分的小标题,文字应简 明扼要。目录按章节排列编写,标明页数,便于阅读。章节、小 节分别以一、(一)1.(1)等数字依次标出。要求标题层次清 晰。目录中的标题应与正文中的标题一致。

\subsubsection{标题}
每章标题以三号字黑体居中打印;“章”下空二行为“节”,以四 号黑体左起打印;“节”下空一行为“小节”,以五号黑体左起打印。换行打印 论文(设计)正文。论文(设计)题目应该简短、明确、有概括性;字数 要适当,一般不宜超过20个汉字。

\subsection{正文格式要求}
\subsubsection{正文}
采用五号宋体字打印。正文是对研究工作的详细表述,一般由标题、文字、 图、表格和公式等部分组成。
\subsubsection{图}
图题若采用中英文对照时,其英文字体为五号正体,中文字体为五号 楷体。引用图应在图题的左上角标出文献来源;图号按章顺序编号,如图3-1为 第三章第一图。如图含有几个不同部分应将分图号标注在分图的左上角,并在图 题下列出各部分内容。\par
一张图片大概这样,简单如图 \ref{fig:hanhan} 所示。
\begin{figure}[!ht]
  \centering
  \includegraphics[width=0.45\linewidth]{图片1.jpg}
  \caption{实验数据的图片}
  \label{fig:hanhan}
\end{figure} 
多张图片,网上也有很多方法。
% 简单如图 \ref{fig:hanhan1} 所示。
% \begin{figure}
%   \centering
%   \includegraphics[width=0.45\linewidth]{图片1.jpg}
%   \includegraphics[width=0.45\linewidth]{图片1.jpg}
%   \includegraphics[width=0.45\linewidth]{图片1.jpg}
%   \includegraphics[width=0.45\linewidth]{图片1.jpg}
%   \caption{实验数据的图片}
%   \label{fig:hanhan1}
% \end{figure} 
\subsubsection{表格}
表格按章顺序编号,如表3-1为第三章第一表。表应有标题,表内必须按规定的符号注明单位。
对表格来说并排表格既可以像图~\ref{tab:parallel1}、
表\ref{tab:parallel2} 使用小页环境,也可以如表\ref{tab:subtable} 使用子表格来做。R语言可以使用xtable程序包直接导出表格代码,复制粘贴一下就ok。

\begin{table}[htbp]
\noindent\begin{minipage}{0.5\textwidth}
\centering
\caption{第一个并排子表格}
\label{tab:parallel1}
\begin{tabular}{p{2cm}p{2cm}}
\toprule
111 & 222 \\\midrule
222 & 333 \\\bottomrule
\end{tabular}
\end{minipage}%
\begin{minipage}{0.5\textwidth}
\centering
\caption{第二个并排子表格}
\label{tab:parallel2}
\begin{tabular}{p{2cm}p{2cm}}
\toprule
111 & 222 \\\midrule
222 & 333 \\\bottomrule
\end{tabular}
\end{minipage}
\end{table}


\begin{table}[htbp]
\centering
\caption{并排子表格}
\label{tab:subtable}
\subcaptionbox{第一个子表格}
{
\begin{tabular}{c|c|c}
\hline
模型 & 得分1 &得分2 \\
\hline
a &111 & 222 \\
b &222 & 333 \\
\hline
\end{tabular}
}
\hskip2cm
\subcaptionbox{第二个子表格}
{
\begin{tabular}{c|c|c}
\hline
模型 & 得分1 &得分2 \\
\hline
a &111 & 222 \\
b &222 & 333 \\
\hline
\end{tabular}
}
\end{table}


\subsubsection{公式}
公式书写应在文中另起一行。并使用英文标点。
考虑模型\eqref{eq1}
\begin{equation}
y_i = \mu(t_i) + e_i, \quad i = 1,2,\dots n,
\label{eq1}
\end{equation}
如果需要换行
\begin{equation}
  \begin{aligned}
  y_i &= \mu(t_i) + e_i\\
      &= \mu(t_i) + e_i.\\
  \end{aligned}
\end{equation}

\subsubsection{参考文献}
毕业论文(设计)不可缺少的组成部分,也是作者对他人知识
成果的承认和尊重。参考文献一律列于文末。连续出版物(期刊):序号 作者,题名刊名,出版年,期号,起止 日;
专著:序号 作者,书名,版本(第1版不标注),出版地,出版年, 起止页码; 论文集:序号,作者,题名,主编,论文集名,出版地,出版年,起 止页码;
学位论文:序号 题名;【学位论文】(英文用【Dissertation】),
保存地点,保存单位,年份。\par
使用方法:在导出bib格式文献引用,将内容复制到main.bib,之后直接使用“cite”,如文献中指出\cite{guegan2005can},\citet{guegan2005can}和\citet{dahlhaus1997}指出,文献中指出\cite{guegan2005can,dahlhaus1997}等用法。


\section{实证分析}
%S&P 500 daily returns. 
\subsection{数据介绍}
介绍数据背景相关文献,表格图片计算结果以及结合实际的深刻分析。

% \begin{figure}
%   \centering
%   \includegraphics[width=0.45\textwidth]{data/IBM_LR.png}
%   \includegraphics[width=0.45\textwidth]{data/IBM_SLR.png}
%   \caption{左图:IBM股票从2005年7月14日到2013年8月30日的对数收益率;右图:IBM股票从2005年7月14日到2013年8月30日的平方对数收益率}
% \end{figure}

\subsection{数据分析}


\section{结论与局限性}
\subsection{结论}
总结全文
\subsection{局限性}
不足和未来发展方向

\newpage

\bibstyle{gbt7714-numerical}
\bibliography{main}
 
\section{附录}
一些不想放在正文的细节。不宜放在正文中,但有参考价值的内容。如调查问卷、 公式推演、编写程序、原始数据附表等。
\subsection{核心代码}
\subsubsection{检验统计量计算}
可以直接从文件中导入代码。见注释。
% \lstinputlisting[language=r]{Statistic.R}


\statement %声明页
\hfill 2020年5月8日 %日期

\end{document}
