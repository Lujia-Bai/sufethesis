\documentclass{SUFEThesis}


\sufesetup{
    title = {上海财经大学 \LaTeX 模板},
    author = {白露佳},
    id = {2016110226},
    school ={统计与管理学院},
    discipline = {统计学},
    supervisor = {柏杨}
} 

\begin{document}

\ctexset{
  section = {
    name={,、 },number={\chinese{section}},format={\centering\heiti\zihao{3}},afterskip={2ex},beforeskip={2ex}, aftername={}
  },%出来是 一、标题
  subsection = {
    name={(,)},number={\chinese{subsection}},format={\heiti\zihao{4}},afterskip={2ex},beforeskip={2ex},aftername={ }
  },%出来是 (一) 标题
  subsubsection = {
    name={,.},number=\arabic{subsubsection},format={\heiti\zihao{5}},afterskip={2ex},beforeskip={2ex},aftername={ }
  }%出来是 1.标题
}

\maketitle

% !TeX root = ../main.tex
\begin{abstract}
论文的摘要是对论文研究内容和成果的高度概括。摘要应对论文所研究的问题及其研究目
的进行描述,对研究方法和过程进行简单介绍,对研究成果和所得结论进行概括。摘要应
具有独立性和自明性,其内容应包含与论文全文同等量的主要信息。使读者即使不阅读全
文,通过摘要就能了解论文的总体内容和主要成果。摘要以浓缩的形式概括研究课题的内容,中文摘要在300汉字左右,英文摘要应与中文摘要基本相对应。

本文介绍上海财经大学论文模板SUFEThesis的使用方法。本模板符合学校的本科论文格式要求。

本文的创新点主要有:
\begin{itemize}
  \item 简单易懂代码少
  \item 用2016本科毕业论文要求里的内容填充无关紧要的部分;
  \item 一边学习摸索一边编写新代码。
\end{itemize}

关键词是为了文献标引工作、用以表示全文主要内容信息的单词或术语。关键词不超过 6
个,每个关键词中间用分号分隔。(模板作者注:关键词分隔符不用考虑,模板会自动处
理。英文关键词同理。)
\sufesetup{
keywords = {毕业论文, 排版, \LaTeX, \TeX, CJK},
}
\end{abstract} 


\begin{abstract*}
An abstract of a dissertation is a summary and extraction of research work
and contributions. Included in an abstract should be description of research
topic and research objective, brief introduction to methodology and research
process, and summarization of conclusion and contributions of the
research. An abstract should be characterized by independence and clarity and
carry identical information with the dissertation. It should be such that the
general idea and major contributions of the dissertation are conveyed without
reading the dissertation.

An abstract should be concise and to the point. It is a misunderstanding to
make an abstract an outline of the dissertation and words ``the first
chapter'', ``the second chapter'' and the like should be avoided in the
abstract.

Key words are terms used in a dissertation for indexing, reflecting core
information of the dissertation. An abstract may contain a maximum of 5 key
words, with semi-colons used in between to separate one another.
\sufesetup{
keywords* = {thesis, template, \LaTeX, \TeX, CJK },
}
\end{abstract*}

\thispagestyle{empty}
\begin{center}
\zihao{3} \LaTeX Template for SHUFE
\end{center}

\begin{center}
\zihao{3}{Abstract}
\end{center}
 
\begin{adjustwidth}{1cm}{1cm}
\hspace{1.5em}
An abstract of a dissertation is a summary and extraction of research work
and contributions. Included in an abstract should be description of research
topic and research objective, brief introduction to methodology and research
process, and summarization of conclusion and contributions of the
research. An abstract should be characterized by independence and clarity and
carry identical information with the dissertation. It should be such that the
general idea and major contributions of the dissertation are conveyed without
reading the dissertation.

An abstract should be concise and to the point. It is a misunderstanding to
make an abstract an outline of the dissertation and words ``the first
chapter'', ``the second chapter'' and the like should be avoided in the
abstract.

Key words are terms used in a dissertation for indexing, reflecting core
information of the dissertation. An abstract may contain a maximum of 5 key
words, with semi-colons used in between to separate one another.\\
\\
\textbf{Key words:} thesis, template, \LaTeX, \TeX, CJK 
\end{adjustwidth}

\newpage

\thispagestyle{empty}
\zihao{5}
\heiti
\tableofcontents
\newpage
\songti


\setcounter{page}{1}
\section{前言}
\subsection{研究背景及意义}

\subsection{研究现状与评价}


\subsection{本文贡献与创新点}

\subsection{文章基本思路与结构}


\section{检验方法概述}

\subsection{参数检验方法}

\subsubsection{检验方法一}
\subsubsection{检验方法二}

\subsection{非参数检验方法}

\subsubsection{检验方法一}
\subsubsection{检验方法二}

\section{数值模拟}
%%%%size
%rho theta
%non-stationarity tvFARIMA: 1st order 2nd order with trend
%T
\subsection{模型介绍}
考虑模型\eqref{eq1}
\begin{equation}
y_i = \mu(t_i) + e_i, \quad i = 1,2,\dots n,
\label{eq1}
\end{equation}
如果需要换行
\begin{equation}
  \begin{aligned}
  y_i &= \mu(t_i) + e_i\\
      &= \mu(t_i) + e_i\\
  \end{aligned}
\end{equation}


\subsection{构建指标体系}


\subsection{模拟过程及结果分析}
数据图片表格之外还要有分析!不会用就去百度BING一下,这里介绍一些简单操作。
\subsubsection{图片}
一张图片,网上也有其他方法(不喜勿喷)
\begin{figure}[!ht]
  \centering
  \includegraphics[width=0.45\linewidth]{图片1.jpg}
  \caption{实验数据的图片}
\end{figure} 
多张图片,网上也有其他方法(不喜勿喷),图 \ref{fig:hanhan} 所示。
\begin{figure}
  \centering
  \includegraphics[width=0.45\linewidth]{图片1.jpg}
  \includegraphics[width=0.45\linewidth]{图片1.jpg}
  \includegraphics[width=0.45\linewidth]{图片1.jpg}
  \includegraphics[width=0.45\linewidth]{图片1.jpg}
  \caption{实验数据的图片}
  \label{fig:hanhan}
\end{figure} 


\section{实证分析}
%S&P 500 daily returns. 
\subsection{数据介绍}
介绍数据背景相关文献,表格图片计算结果以及结合实际的深刻分析。

% \begin{figure}
%   \centering
%   \includegraphics[width=0.45\textwidth]{data/IBM_LR.png}
%   \includegraphics[width=0.45\textwidth]{data/IBM_SLR.png}
%   \caption{左图:IBM股票从2005年7月14日到2013年8月30日的对数收益率;右图:IBM股票从2005年7月14日到2013年8月30日的平方对数收益率}
% \end{figure}

\subsection{数据分析}


\section{结论与局限性}
\subsection{结论}
总结全文
\subsection{局限性}
不足和未来发展方向

\newpage
\bibliographystyle{gbt7714-numerical}
\bibliography{main}
 
\section{附录}
一些不想放在正文的细节。
\subsection{核心代码}
\subsubsection{检验统计量计算}
% \lstinputlisting[language=r]{Statistic.R}

% \includepdf[pages={1}]{声明.pdf} 


\end{document}
