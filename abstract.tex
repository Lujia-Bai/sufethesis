% !TeX root = ../main.tex
\begin{abstract}
 论文的摘要是对论文研究内容和成果的高度概括。摘要应对论文所研究的问题及其研究目
的进行描述,对研究方法和过程进行简单介绍,对研究成果和所得结论进行概括。摘要应
具有独立性和自明性,其内容应包含与论文全文同等量的主要信息。使读者即使不阅读全
文,通过摘要就能了解论文的总体内容和主要成果。

论文摘要的书写应力求精确、简明。切忌写成对论文书写内容进行提要的形式,尤其要避
  免“第 1 章……;第 2 章……;……”这种或类似的陈述方式。

本文介绍上海财经大学论文模板SUFEThesis的使用方法。本模板符合学校的本科论文格式要求。

本文的创新点主要有:
\begin{itemize}
  \item 简单易懂代码少
  \item 用ThuThesis里的废话填充无关紧要的部分;
  \item 一边学习摸索一边编写新代码。
\end{itemize}

关键词是为了文献标引工作、用以表示全文主要内容信息的单词或术语。关键词不超过 5
个,每个关键词中间用分号分隔。(模板作者注:关键词分隔符不用考虑,模板会自动处
理。英文关键词同理。)
%%%%%%%%%%%%%%%%%%%%%%%%%%%%%%%%%%%%%%%%%%%%
\\%不要动 要有空行
\\%不要动 要有空行
%%%%%%%%%%%%%%%%%%%%%%%%%%%%%%%%%%%%%%%%%%
\sufesetup{
keywords = {毕业论文, 排版, \LaTeX, \TeX, CJK},
}
\end{abstract}